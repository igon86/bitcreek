\section{Prefazione}

L'obbiettivo del progetto finale era realizzare un sistema P2P conforme alle specifiche di BitTorrent per scopi didattici utilizzando un paradigma di programmazione orientato agli oggetti (linguaggio Java).
Possiamo affermare di aver raggiunto questo obbiettivo realizzando un sistema funzionale ed efficiente, conforme alle specifiche date e con features/politiche aggiuntive derivate direttamente da quelle del protocollo e dei client BitTorrent attuali.
Grazie ad una buona fase iniziale di progettazione delle varie classi la stesura del codice non \`e risultata eccessivamente complessa, come framework abbiamo utilizzato NetBeans IDE sia per la strutturazione del codice java che dei vari diagrammi UML; dopo una prima stesura del codice effettuata "a due mani" abbiamo utilizzato Subversion per la gestione delle varie revisioni del codice e la loro condivisione. Abbiamo sfruttato tutte le nostre conoscenze acquisite durante il corso per l'implementazione delle varie parti del progetto avendo cura di sfruttare appieno il paradigma di programmazione ad oggetti e le potenzialit\`a del linguaggio Java.
\linebreak
Gran parte dei concetti e delle direttive presenti nella bozza di progetto sono stati rielaborati accuratamente in sede di progetto, la scelta delle politche da utilizzare \`e derivata in gran parte dalle reference aggiuntive presenti nella bozza mentre per la realizzazione della GUI abbiamo preso a riferimento client BitTorrent attuali (Azureus nel nostro caso). 
\linebreak
Altre features aggiuntive sono state implementate per rendere il nostro software utilizzabile in un contesto reale, in particolare abbiamo posto grande attenzione nel realizzare un interfaccia grafica il pi\`u possibile reattiva e comprensibile, abbiamo aggiunto meccanismi per il salvataggio su file (.creek) dei riferimenti allo swarm, meccanismi di terminazione gentile sia del client che del server con salvataggio dello stato, aggiramento di NAT e firewall, risolto problematiche relative al trasferimento di file di grosse dimensioni.
Grazie a queste features aggiuntive ci sentiamo di affermare che il nostro programma \`e utilizzabile anche in un contesto reale.
\linebreak
Tutti i vari aspetti qui trattati verrano adeguatamente spiegati nel resto della relazione, sia come analisi dei singoli package sia come spiegazione delle varie features e politiche implementate.